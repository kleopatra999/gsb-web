%% LyX 1.5.1 created this file.  For more info, see http://www.lyx.org/.
%% Do not edit unless you really know what you are doing.
\documentclass[12pt,english]{scrartcl}
\usepackage[T1]{fontenc}
\usepackage[latin9]{inputenc}
\IfFileExists{url.sty}{\usepackage{url}}
                      {\newcommand{\url}{\texttt}}

\makeatletter

%%%%%%%%%%%%%%%%%%%%%%%%%%%%%% LyX specific LaTeX commands.
\providecommand{\LyX}{L\kern-.1667em\lower.25em\hbox{Y}\kern-.125emX\@}

%%%%%%%%%%%%%%%%%%%%%%%%%%%%%% Textclass specific LaTeX commands.
\newenvironment{lyxcode}
{\begin{list}{}{
\setlength{\rightmargin}{\leftmargin}
\setlength{\listparindent}{0pt}% needed for AMS classes
\raggedright
\setlength{\itemsep}{0pt}
\setlength{\parsep}{0pt}
\normalfont\ttfamily}%
 \item[]}
{\end{list}}

%%%%%%%%%%%%%%%%%%%%%%%%%%%%%% User specified LaTeX commands.
%\usepackage{hyperref}

% provides missing characters,
% see note in chapter 'Character Tables'
%\usepackage{textcomp}

\usepackage{ifpdf} % part of the hyperref bundle
\ifpdf % if pdflatex is used

% set fonts for nicer pdf view
\IfFileExists{lmodern.sty}{\usepackage{lmodern}}{%
\usepackage[scaled=0.92]{helvet}
\usepackage{mathptmx}
\usepackage{courier} }
% the pages of the TOC are numbered roman
% and a pdf-bookmark for the TOC is added
\pagenumbering{roman}
\let\myTOC\tableofcontents
\renewcommand\tableofcontents{%
\pdfbookmark[1]{Contents}{}
\myTOC
\clearpage
\pagenumbering{arabic} }
% link all cross references and URLs in pdf output
\usepackage[colorlinks=true, bookmarks, bookmarksnumbered,
linkcolor=black, citecolor=black, urlcolor=blue, filecolor=blue,
pdfpagelayout=OneColumn, pdfnewwindow=true,
pdfstartview=XYZ, plainpages=false, pdfpagelabels,
pdfauthor={Chip Cuccio}, pdftex,
pdfkeywords={}]{hyperref}
\else % if dvi or ps is produced

% link all cross references and URLs in dvi output
\usepackage[ps2pdf]{hyperref}

% the pages of the TOC are numbered roman
\pagenumbering{roman}
\let\myTOC\tableofcontents
\renewcommand\tableofcontents{%
\pdfbookmark[1]{Contents}{}
\myTOC
\clearpage
\pagenumbering{arabic} }
\fi 

\usepackage{babel}
\makeatother

\begin{document}

\title{GNOME SlackBuild (GSB) Website Documentation}


\author{Chip Cuccio }


\date{11/24/2007}

\maketitle
\begin{abstract}
This document explains how the GNOME SlackBuild (GSB) website was
developed, how it works, how to maintain it, etc. This document also
contains other general site-related information for developers of
the GSB project, as well as web/sysadmins assigned web-related duties.
This document is {}``living'' - in other words, a work-in-progress.
\end{abstract}
\tableofcontents{}\newpage{}


\section{Introduction}

The GNOME SlackBuild (GSB) website was initially developed by Chip
Cuccio in early 2005 as a {}``static'' website. However, it didn't
take long before we realized that we needed the site to be a bit more
dynamic. At the same token, I really didn't want to deploy (or write)
a CMS-type of app that used some DB back-end. I wanted tiny, fast,
and simple. Really, the only key points of the website that are dynamic,
are the news, version numbers, and changelogs, and the project TODOs.

So I custom-wrote a tiny {}``CMS'' that uses flat files as its {}``DB''.
It's tiny, portable (and has been moved around from server-to-server
a bunch!), and is simple for anyone to update content. Since 2005,
it's the same code that's been in use (with a bunch of updates and
revisions), and it still works well to this day,


\section{Why this document?}

This document was written to help developers or web/sysadmins manage,
maintain, and continue the active development of the GSB website.
Because the site was custom-written, a good document explaining how
everything works is necessary.


\section{How the site works}


\subsection{Basic technical info}

The GSB site's main {}``CMS engine'' is written in PHP. The engine
performs parsing/'includes' against basic HTML templates for the content/copy
in the main sections of the website. There are some support scripts
and programs, not directly related to content rendering, written in
shell and PHP as well.

The GSB site was written with strict adherence to standards, as well
as, with excellent accessibility and usability guidelines. The site
renders it's output in XHTML 1.1 (modular), with MathML plus SVG extensions.
Styling is handled by strict CSS 2.0, with no {}``IE hacks''. Visitors
of the site whom use IE will see the site just fine, but will not
be graced with the styling the CSS renders. The site works perfectly
in modern standards-compliant graphical browsers, as well as older
legacy browsers. Additionally, the XHTML output renders beautifully
in text-based browsers (`lynx`, etc.) and screen-readers. I designed
the site intentionally, with forward-looking standards and structure,
but at the same token, it degrades nicely in non-graphical and legacy
environments. Again, IE will see the site in a degraded text-only
mode, simply because IE sucks ass, and I refuse to write CSS hacks
so that it looks pretty in a shitty browser.

It is important to note, that since the site is rendered in XHTML
1.1, the content-type on the server side, serves the pages as {}``application/xhtml+xml''.
This creates even more strict-ness when writing web code - since one
tiny malformation or syntax error will render the entire page unreadable
by many browsers. \emph{All pages, all the time, should validate as
100\% pure, valid XHTML 1.1} - always!

The chosen {}``LANG'' (or {}``charset'') attribute used to serve
the site, is UTF-8. No other character set will do. UTF-8/Unicode
is great stuff, and it's the future.

All images on the site are in PNG format. No GIFs, no JPEGs, no TIFFs.
Just PNGs. Larger documents are in plain text, PDF, or just HTML.
Documents which were written in \LaTeX{} or other typesetting apps,
then rendered to other formats, must include the original source code
in SVN. Even the images I've made have the source PSD files committed
to SVN. Another good example is this very document: although it was
exported to PDF, etc., it was written in \LyX{}/\LaTeX{} - hence,
the source files are in SVN. \emph{Nothing in this site is closed-source}.
Ever. It's all in SVN for public review and consumption.


\subsubsection{Code review}

It is highly recommended that anyone who wishes to help with the website,
'checkout' the web code from SVN and perform a thorough code review.
If you have a web server accessible locally, you can even use the
GSB site on that to debug, work on, test, etc.


\subsection{Style guide}

I'll try and keep this short and simple. But proper structure and
semantics is of utmost importance when maintaining the GSB site. Here
is the basic style guide I've used when writing/maintaining the GSB
site:

\begin{itemize}
\item <h1> tags are reserved for the main site title and header. \emph{They
cannot be used anywhere else}, and no one should have to modify the
pre-defined <h1> header.
\item <h2> tags are reserved for page and (main) section headers. Every
single page will have at least one <h2> tag. These can be modified
and/or defined.
\item <h3> through <h6> tags are used as sub-sections and nested sub-sections,
and \emph{must} have a {}``parent'' tag/section above it. For example,
you cannot use an <h3> tag for a page or main section header. Since
<h2> tags are reserved for page and/or main section headers, <h3>
must be structurally below <h2>, and so on. For the news section,
news article titles are always enclosed in <h3> tags - and it's automatically
done by the news rendering engine (more on that later).
\item Ordered lists vs. unordered lists:\\
Ordered lists are normally used where chronology is important, or
when prefacing text specifies an exact number of forthcoming items.
If the intended content constitutes a list, \emph{use a list}. Of
course, when appropriate, lists can be nested - even mixed nests (both
ordered and unordered).
\item Paragraphs: \\
Any stand-alone body of text should be enclosed in <p> tags. Line-breaks
(<br />) should be used conservatively. The use of <div> tags to enclose
stand-alone blocks of text is strongly discouraged. If text within
a paragraph or other block-level elements needs markup or styling,
use <span> tags with class attributes.
\item Definition lists: \\
Definition lists should be used when specifying and defining terms,
or when specifying figurative questions and providing answers (such
as our FAQ on the website).
\item Code and pre-formatted text:\\
All literal code must be enclosed in <code> and <pre> tags. Short
snippets of code should be enclosed in <code> tags, while larger blocks
of code should be enclosed in <pre> tags. \emph{Do not} enclose <pre>
blocks of text within any other block-level element.
\item HTML Entities:\\
Escape HTML entities! For example, {}``<'' must be coded as \&lt;
- otherwise the page will be malformed and will not load in many user
agents.
\item URIs:\\
Links should never be verbs (e.g.: no {}``click here''). Literal
URIs, whether linked or unlinked should be enclosed within brackets
- e.g.: <http://gnomeslackbuild.org>.
\item Quotes:\\
Short quotes should always be wrapped in <q> tags. Never wrap quotes
in quotation marks when using the <q> tag! For larger blocks of quotes,
enclose them in <blockquote> tags. Additional block-level elements
are permitted and within <blockquote> elements.

\begin{itemize}
\item Citing:\\
When using <q> and/or <blockquote> tags and quoting something/someone,
if possible, utilize the 'cite' attribute within the tag.
\end{itemize}
\item Tables should only be used to present \emph{tabular data}. They are
never to be used to handle/render layout. Ever!
\end{itemize}

\subsection{Site layout}

The layout of the site is quite simple:

\begin{itemize}
\item Homepage
\item Sections

\begin{itemize}
\item Subsections
\end{itemize}
\end{itemize}
And that's about it. I try to keep sections only two levels deep maximum,
so not to lose folks. There's no search facility on the site, because
it's so small, and because Google can do it anyway.


\section{Helping with the site}

I won't get into detail as to how the site's CMS engine works, because
as a web hacker, it's your job to figure it out. Folks not familiar
with PHP/HTML/CSS/etc. should not express interest in maintaining
the site - 'cause you'll just be denied. We need experienced folks
to help.

But I will cover a few basic, yet important aspect of the site, where
maintaining content and a few other things is ridiculously easy.


\subsection{News}

On the landing/home page, there's a news section (and it's also in
it's own main section - for archival purposes). All of those articles
are handled and maintained in a flat text file within the <docroot>
of the site: {}``news.txt''. News is displayed on the site in reverse-chronological
order (newest entry first). It's very simple to edit/add news; by
editing that one small news.txt file.

The file is a specially-formatted text file, and is parsed by some
PHP functions to display nice and cleanly on the site.

The format of the news.txt file is as follows:

\begin{lyxcode}
\#\#\#

date-time

title~of~news~entry

body~of~news~entry
\end{lyxcode}
The \char`\"{}\#\#\#\char`\"{} text, is actually a separator, so that
the PHP news function knows to parse that as one entire \char`\"{}article\char`\"{}.
The body section of the news entry accepts HTML, but that is the only
place HTML is used. URIs are automatically made as links. To reference
an internal URI, one must use the following format:

\begin{lyxcode}
<a~href=\char`\"{}/foo/\char`\"{}>See~the~foo~section</a>
\end{lyxcode}
Hint: Internal URIs /never/ use \char`\"{}http://\char`\"{} - ever.
Reference them to the base/root of the site, as shown in the above
example.

Just open the <svnroot>/web/news.txt file for actual live examples
used on the site.


\subsubsection{News style guide}

The news articles actually have their own style, and it's consistent.
It is important to note, that news articles are automatically enclosed
within HTML block-level elements. However, you still must enclose
blocks of text/paragraphs in block-level elements (mainly, \char`\"{}<p>\char`\"{}).
Lists, are OK - but header tags are not. The site news engine will
handle everything else for you.


\subsection{Screenshots}

Screenshots are placed in \texttt{`<webroot>/screenies`}. Then, a
simple shell script takes care of resizing, creating thumbnails, and
adding eveything to the screenshots page. After screenshots have been
dropped into the proper directory run:

\begin{lyxcode}
cd~../support-scripts

./make-GSB-thumbs.sh
\end{lyxcode}
And that's it. Commit the changes to SVN, sync to staging, and make
sure the screenshots page/section looks and works OK (i.e.: New thumbnails
should now appear on the page. Thumbnails should be clickable links
to full-size images. Full-size images should be in the ballpack of
\textasciitilde{}800x600 resolution.)


\subsection{Version numbers}

Version numbers are used all over the site for our project. But they
only need to be maintained/specified in one tiny file:\\
<svnroot>/web/versions.txt

The legend for the version numbers is explained in the header of that
file. To reference version numbers in the site templates ({}``<svnroot>/web/content/foo.html\char`\"{}):

\begin{itemize}
\item \texttt{<?php echo \$gsb\_bin\_stable\_ver; ?>}\\
References the GSB actual stable binary release (tracks GNOME ver.)
\item \texttt{<?php echo \$gsb\_source\_ver; ?>}\\
References the GSB source release. ...and so on (see <svnroot>web/common/versions\_inc.php
for all variables).
\end{itemize}

\subsection{Getting SVN commit access to the site}

We don't just let any Tom Dick and Harry have SVN commit access to
the site. You must prove yourself first by submitting patches, etc.
Once we've learned to trust you and your work, you'll likely be granted
SVN commit access to maintain the site.

Developing/maintaining the GSB website is a three step process:

\begin{enumerate}
\item Develop and commit to <svnroot>/web
\item Sync the <svnroot>/web code to the staging site for testing, review,
and approval.
\item Migrate staging to production
\end{enumerate}
Under no circumstances is the staging environment to be used as the
development environment. Development occurs on your own server/workstations.
Staging is where we test and approve before pushing to production.

Even to migrate the SVN web code to the staging site, you must be
granted access.

To migrate SVN web code to the staging site, we use a web-based facility
(written by yours truly): <\url{http://admin.gnomeslackbuild.org}>.
It's locked down, and web developers we can trust will be granted
access to the facility.

At the moment, code migrations to the production site are handled
by me once it's been approved. I'm very picky about the site functionality
and the look-n-feel, it's just the way it works for now.


\section{Open to suggestions}

I should close saying that I'm always open to suggestions. I may dismiss
most of them, though. All kidding aside, I embrace ideas, feedback,
and web/code hackers to help make and keep this site as cool as the
project it hosts. Thanks.
\end{document}
